\documentclass[]{article}
\usepackage[czech]{babel}
\usepackage[utf8]{inputenc}

\begin{document}

\title{Kapitola 1: Definice základních pojmů}
\author{Bc. Štěpán Heller}
\date{\today}
\maketitle

\section{Motivace k řízení podnikových procesů}
Každá firma, která se snaží efektivně řídit svůj chod a neustále se rozvíjet stále hledá nové a nové cesty, jak toho docílit. Takovými cestami může být uvádění nových produktů na trh, hledání nových trhů a příležitostí na nich, nabírání nových zaměstnanců, investice do propracovaného marketingu a mnoho dalších. Stále více firem ale v posledních několika dekádách obrací svou pozornost také dovnitř vlastní organizace. Hledají oblasti, kde je možné najít úspory nebo kde by bylo možné práci zefektivnit.

Aby bylo něco takového vůbec možné, musí mít manažeři a odpovědní vedoucí pracovníci především přehled o své organizaci a její hlubokou znalost. Pouze z takové hluboké znalosti pak mohou vzejít příležitosti k efektivnějšímu dosahování podnikových cílů.

Z toho důvodu firmy a organizace hledají cesty, jak lépe pochopit a následně standardizovat \textit{procesy} v rámci vlastního podniku, které ve svém souhrnu nejsou nic jiného než soubor postupů, kterými podnik nebo organizace dosahuje svých cílů. O tom, kterak takové procesy pozorovat, standardizovat a řídit, byla napsána celá řada publikací a je nezbytné mít na zřeteli, že se jednotlivé přístupy od sebe více či méně odlišují. 

Ačkoliv akademici i odpovědní lidé z prostředí samotných firem a organizací se často přou, který z přístupů je lepší, zůstává bez nejmenších pochyb, že adoptování jakéhokoliv přístupu vedoucího k lepšímu porozumění chodu vlastní organizace je lepší, než nahodilý přístup k řízení společnosti, kdy jsou změny vykonávany ad hoc a jakékoliv plánování do budoucna je tak velmi obtížné. Pochopení vlastní organizace je jedním (nikoli jediným) ze základních předpokladů pro její dlouhodbě udržitelný rozvoj a právě k tomu jsou procesy a procesní řízení široce akceptovaným přístupem.

The main objective of process management is to find the most efficient and effective way to transform customer requirements into customer satisfaction.

\section{Definice základních pojmů}
V rámci této sekce jsou definovány základní pojmy, jejichž znalost a plné porozumění je nezbytné pro orientaci v obsahu dalších kapitol.
\subsection{Podnikový proces}
V nadsázce řečeno definic pojmu \textit{proces} existuje tolik, kolik existuje publikací, které jsou jim věnovány. Intuitivně si člověk s těmito definicemi neseznámený představí určitou po sobě jdoucí posloupnost operací na jejichž konci může být nějaký výsledek.

Norma EN ISO 9000:2000 definuje pojem proces následovně: \cite{iso_9000}

\begin{quote}
Proces je soubor vzájemně působících nebo vzájemně souvisejících činností, které přeměňují vstupy na výstupy.
\footnote{Set of interrelated or interacting activities which transforms inputs into outputs.}
\end{quote}

O něco podrobnější definici můžeme najít v \cite{Weske2007}:

\begin{quote}
Podnikový proces se skládá ze souboru činností, které jsou prováděny koordinovaně v organizačním a technickém prostředí. Tyto činnosti společně plní podnikový cíl. Každý podnikový proces je prováděn jednou organizací, ale může vzájemně působit s procesy prováděnými jinými organizacemi.
\footnote{A business process consists of a set of activities that are performed in coordination in an organizational and technical environment. These activities jointly realize a business goal. Each business process is enacted by a single organization, but it may interact with business processes performed by other organizations. \cite{Weske2007}}
\end{quote}

I když je možné proces vnímat jako izolovanou jednotku, je na tomto místě dobré si uvědomit, že proces v organizaci je velmi často výstupem jiného procesu. Mezi hlavní atributy procesu patří: \cite{Bandor2007}

\begin{itemize}
\item \textit{Název}, který proces identifikuje.
\item \textit{Účel}, pro který je proces prováděn.
\item \textit{Vlastník}, který je za proces zodpovědný (osoba nebo složka v organizaci).
\item \textit{Specifikace vstupů}. Věci, které jsou potřebné k provádění procesu.
\item \textit{Specifikace výstupů}. Věci, které jsou vytvořeny v průběhu provádění procesu.
\item \textit{Vstupní a výstupní podmínky}, které musí být splněny při spuštění a ukočení procesu.
\item \textit{Činnosti} definují jednotlivé kroky (operace) při provádění procesu.
\item \textit{Role a zodpovědnosti} definují, kdo je zodpovědný za provedení konkrétní činnosti.
\end{itemize}

\subsubsection{Produkt}
Dle \cite{iso_9000} je produkt definován jednoduše jako výsledek procesu. Produkt se dále může skládat z dalších produktů, které jsou výsledkem činnosti jiných procesů.

\subsubsection{Procedura}
Proceduru definujeme dle \cite{iso_9000} jako \textit{\uv{určený způsob, jak vykonat činnost nebo proces}}.
\footnote{Specified way to carry out an activity or a process.\cite{iso_9000}}

Pro správné pochopení rozdílu mezi procesem a procedurou je třeba si uvědomit, že proces nám říká \uv{co je potřeba udělat, a které role jsou zastoupeny} a procedura \uv{jak to udělat} a většinou se týká pouze jedné role. \cite{Bandor2007}


\subsection{Řízení podnikových procesů}
The main objective of process management is to find the most efficient and effective way to transform customer requirements into customer satisfaction. \cite{Jedlitschka2010}
\subsection{Business Process Management System}
\subsection{Capability Maturity Model}
\subsection{Business Process Model}
\subsection{Business Process Lifecycle}
\subsection{Classification of Business Processes}

\section{Vývoj přístupů k řízení podnikových procesů}

\bibliographystyle{plain}
\bibliography{Bibliography}

\end{document}