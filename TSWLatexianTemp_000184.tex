\documentclass[]{article}
\usepackage[czech]{babel}
\usepackage[utf8]{inputenc}
\usepackage{float}
\usepackage{graphicx}

\newcommand{\ptheory}{$\Psi$-theory}

\begin{document}

\title{Kapitola 4: Metodologie DEMO}
\author{Bc. Štěpán Heller}
\date{\today}
\maketitle

\section{O DEMO}
\subsection{Rozdíl mezi notací a metodologií}
V celém textu se objevují 2 jevy. O BPMN mluvíme jako o notaci a o DEMO jako o metodologii. Takto tyto techniky označujeme záměrně a myslím, že je na tomto místě účelné si vysvětlit rozdíl mezi pojmy \textit{notace} a \textit{metodologie}. Pochopení této odlišnosti totiž vnáší trochu světla k lepšímu pochopení rozdílu mezi BPMN a DEMO.

\subsubsection{Notace}
Notace označuje formální prostředky pro popis reality. Například právě v oblasti analýzy a modelování podnikových procesů je notací sada grafických objektů, které pak používáme pro popsání samotného procesu. Na notaci je obvykle navázána související metodika. \cite{Notace}

Metodikou nazýváme popis pracovního postupu nějaké činnosti, který je více či méně formalizovaný. \cite{Metodika} V případě modelovací techniky by taková metodika tedy popisovala jak při modelování postupovat a jak a kdy jednotlivé elementy přesně používat. To však v případě BPMN neplatí, jelikož BPMN není svázané s žádnou metodikou \cite{Vasicek2008} a tedy pokud se někde vyskytuje označení BPMN jako metodiky, je tato formulace chybná.

\subsubsection{Metodologie}
Metodologie je oproti tomu vědní disciplína, která se zabývá tvorbou metod a jejich aplikací. Metodologie vědy je tedy naukou o metodách. Jak píše \cite{Ochrana2009} je teorií k výběru výzkumných metod a návodem, jak vybrané metody (metodu) používat ve vědeckém zkoumání.

Pod vlivem angličtiny se však i v češtině často setkáváme s tím, že pojmy metodika a metodologie splývají a jsou často zaměňovány. Můžeme nicméně vidět, že rozdíl BPMN a DEMO je na první pohled zřejmý v tom, že notace BPMN nemá za sebou zdaleka tak robustní základ jako metodologie DEMO. Tento fakt má několik důsledků týkajících se přístupnosti a přímočarosti použití obou technik. Tyto důsledky budou ještě v této práci dále rozebrány.

\subsection{Motivace k vytvoření DEMO}
Jak shrnuje \cite{Vejrazkova2013}, u základní úvahy tvůrců DEMO byl současný stav podniků a organizací, které jsou velmi komplexní a z toho důvodu je velmi obtížné mít pomocí současných nástrojů jasnou představu o tom, jak přesně fungují a co se v nich děje.

Moderní organizace jsou totiž založeny na propojení sociálních a technických komponent, které spolu vzájemně komunikují. Komunikce je tedy nejdůležitějším aspektem celé metodologie DEMO. Dle \cite{Dietz2006} je ontologie podniků (Enterprise Ontology) nejvhodnějším prostředkem k pochopení konstrukce a operací v podniku.

\begin{quote}
DEMO bylo vytvořeno jako metodologie pro vytváření ontologického modelu podniku. \cite{Vejrazkova2012}
\footnote{DEMO was developed to be a methodology for creating an ontological model of an enterprise. \cite{Vejrazkova2012}}
\end{quote}

\section{Ontologie}

The ontology (or ontological model) of an enterprise is defined as an understanding of its operation, that is completely independent of the realization and the implementation of the enterprise.

   The two system notions
• The teleological system notion
• Is about the function and external behavior of a system
• Is the dominant system concept in the social sciences
• Is perfectly adequate for using and controlling systems
• Has the black-box model as the corresponding kind of model
• The ontological system notion
• Is about the construction and operation of a system
• Is the dominant system concept in the engineering sciences
• Is perfectly adequate for building and changing systems
• Has the white-box model as the corresponding kind of model

By the business of an enterprise is understood the function perspective on the enterprise. It is characterized by the products and services that are delivered to the environment.
A business model of an enterprise is a black-box model type of model
By the organization of an enterprise is understood the construction perspective on the enterprise. It is characterized by the processes in which the products and services are brought about.

Who needs enterprise ontology?
An enterprise ontology provides the common understanding of the operation of an enterprise to all stakeholders. In particular the next groups of stakeholders do need it:
• Managers; managing has become too complex for relying only on the function view on the enterprise. They need to have a global understanding of its operation too.
• Designers; for (re)designing and (re)engineering the organization of an enterprise, an explicit specification of the business processes is needed that is independent of their implementation.
• Users; why should the operation of an enterprise be fully opaque to its users? An enterprise ontology would provide the users the transparency they need too!

\section{Teorie PSI ($\Psi$-theory)}

\begin{quote}
Teorie PSI neboli \ptheory  je teorie o fungování organizací. \cite{Dietz2005}
\footnote{The \ptheory is a theory about the operation of organizations. \cite{Dietz2005}}
\end{quote}

Zkratka PSI znamená \textit{Performance in Social Interaction}. Paradigma, na kterém je tato teorie založena říká, že subjekty, kterými jsou lidé v organizaci, vstupují do závazků a dodržují je. Tímto způsobem pak vzniká spolupráce mezi lidmi. %todo fuj%

Cílem \ptheory je umožnit porozumění funkcím (%todo funkcím?%)
organizace bez vlivu toho, jak jsou tyto funkce ve skutečnosti operativně vykonávány. Jak uvádí \cite{Vejrazkova2013} stejné cíle si klade i metodologie DEMO, takže je jen logické, že je právě na \ptheory postaveno. Porozumění této teorii je tedy nezbytné pro správné pochopení a používání DEMO.

Teorie PSI se skládá ze čtyř axiomů:

\begin{enumerate}
\item operační,
\item transakční,
\item kompoziční,
\item distinkční.
\end{enumerate}

\subsection{Operační axiom – The operation axiom}
První axiom \ptheory se nazývá operační. Jeho základem jsou dvě tvrzení: \cite{Dietz2006}:

\begin{enumerate}
\item The operation of an enterprise is constituted by the activities of actor roles, which are elementary chunks of authority and responsibility, fulfilled by subjects. %todo wtf!!! %
\item Při tom provádějí dva druhy činností: koordinační a produkční (\textit{coordination and production acts}). Výsledkem těchto činností jsou koordinační a produkční skutky (\textit{coordination and production facts}).
\end{enumerate}

Provádět produkční činnost znamená přivádět na svět něco nového a přispívat tak k podnikovým funkcím nebo službám. Produkční skutky pak mohou být hmotné i nehmotné. Příkladem těch hmotných může být například vyrobení pizzy, příkladem nehmotných zase napříkald vynsesení rozsudku soudem.

Provádět koordinační činnost znamená, že subjekty jednají v souladu se závazky k sobě navzájem, které se týkají tvorby produkčních skutků.

Kromě činností a skutků rozlišujeme ještě koordinační a produkční světy. Koordinační svět je možina koordinačních skutků a stejně tak produkční svět je množina produkčních skutků. Oba světy jsou tedy množinou skutků, které byly vytvořeny do konkrétního momentu v čase.

%todo obrázek operačního axiomu%

\subsubsection{Koordinační činnosti}
Koordinační činnost probíhá mezi dvěma subjekty z nichž jeden se nazývá vykonavatel (\textit{performer}) a druhý (\textit{adressee}). Koordinačních činností je několik typů, které můžeme rozdělit na 
c-acts
c-facts
c-world

p-acts
p-facts
p-world
\subsection{Transakční axiom – The transaction axiom}
order phase
execution phase
result phase
\subsection{Kompoziční axiom – The composition axiom}
Such a cluster of transactions is called a business process.
\subsection{Distinkční axiom – The distinction axiom}
\section{Teorém organizace – The organization theorem}
\section{Notace}

\nocite{*}
\bibliographystyle{plain}
\bibliography{Bibliography}

\end{document}