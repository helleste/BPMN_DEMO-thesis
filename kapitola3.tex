\documentclass[]{article}
\usepackage[czech]{babel}
\usepackage[utf8]{inputenc}
\usepackage{float}
\usepackage{graphicx}


\begin{document}

\title{Kapitola 3: Notace BPMN}
\author{Bc. Štěpán Heller}
\date{\today}
\maketitle

\section{O BPMN}
BPMN neboli Business Process Modelling Notation je soubor pravidel a grafických prvků, pomocí kterých mohou organizace modelovat svoje obchodní procesy. Jedná se o světově nejpoužívanější [zdroj?] standard pro modelování obchodních procesů. Jeho nespornou výhodou je, že narozdíl od hojně rozšířené flowchartové techniky, je BPMN standardizované, tudíž je možné modely vytvořené v BPMN například strojově zpracovat. Specifikace BPMN totiž přímo obsahuje definici, jak jednotlivé elementy a vztahy mezi nimi převádět do jazyka BPEL.

Za vznikem BPMN stála iniciativa BPMI (Business Process Management Initiative), jejíž primární motivací bylo vytvořit grafickou notaci, která bude srozumitelná všem účastníkům životního cyklu procesu (management, vývojáři, analytici). \cite{Vasicek2008} 

Dalším cílem, se kterým bylo BPMN vytvořeno bylo ustanovit notaci, které umožní zobrazovat jednoduché i komplexní obchodní procesy \cite{Vasicek2008}, protože v té době obvyklé modelovací metody byly při vytváření rozsáhlých modelů velmi obtížně použitelné a vzniklé modely bylo složité udržovat.

O BPMN se stará organizace Object Management Group (OMG).

\subsection{Verze 1.2 vs 2.0}
BPMN je v současnosti ve verzi 2.0, nicméně v praxi je stále hojně využívána i verze 1.2. 2.0 je někdy mylně pokládána za novější verzi BPMN, nicméně faktem spíše je, že každá z verzí slouží trochu jinému účelu.

Pro práci v obchodním oddělení organizace (například v operacích) se hodí spíše verze 1.2. Oproti tomu v technických divizích, jako je například IT je vhodnější použít 2.0. Je to proto, že OMG doplnila do BPMN 2.0 řadu elementů zaměřených právě na IT oblast. \cite{Panagacos2012} Důvodem byla zejména snazší integrace BPMN do BPM softwaru.

\subsection{Základní elementy notace BPMN}

Symbolika použitá v BPMN je odvozená od klasické a hojně rozšířené flowchartovací techniky, která je intuitivní na pochopení i pro pozorovatele, který není obeznámen s problematikou modelování obchodních procesů.

BPMN obsahuje více jak 100 symbolů. Pro základní práci, ale obvykle stačí jen několik základních symbolů. Základní druhy grafických elementů jsou 4: \cite{Dumas2013}

\begin{enumerate}
\item Flow objects (Tokové objekty)
\item Event (Událost)
\item Activity (Aktivita)
\item Gateway (Brána)
\end{enumerate}

\subsubsection{Flow objects}

\section{BPMN a BPEL}
Standard WS-BPEL popisuje význam a vlastnosti jednotlivých aktivit, jejich XML definici, ale nespecifikuje jejich grafický zápis.
Někteří dodavatelé proto vytvořili svoji vlastní grafickou notaci jazyka BPEL, která se s každou verzí zdokonaluje tak, aby byly procesy co nejpřehlednější.
Naproti tomu BPMN (Business Process Modeling Notation) je grafická notace pro modelování procesů názornou a přehlednou formou.

Cílem BPMN je poskytnout firmám možnost zmapovat procesy a pohlížet na ně v grafické podobě. Přitom podoba a smysl všech symbolů je standardizován,každý přesně rozumí tomu, co daný model znamená a jak proces funguje. To je obrovský přínos, pokud potřebujete procesy komunikovat napříč firmou, mezi odděleními, mezi obchodními a IT útvary.


Rozdíly mezi BPEL a BPMN jsou tedy zřejmé:

BPMN je popisná notace pro business modely procesů a je ustálená na úrovni grafických symbolů, jejich významů a logických propojení.
BPEL je standard pro provádění a popis procesů a hovoří spíše technickou řečí. Je standardizován na úrovni XML definice procesu. \cite{BPELcz} %todo přepsat vlastními slovy%


\nocite{*}
\bibliographystyle{plain}
\bibliography{Bibliography}

\end{document}