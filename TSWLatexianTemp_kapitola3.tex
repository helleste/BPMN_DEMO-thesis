\documentclass[]{article}
\usepackage[czech]{babel}
\usepackage[utf8]{inputenc}
\usepackage{float}
\usepackage{graphicx}


\begin{document}

\title{Kapitola 3: Notace BPMN}
\author{Bc. Štěpán Heller}
\date{\today}
\maketitle

\section{O BPMN}
BPMN neboli Business Process Modelling and Notation je soubor pravidel a grafických prvků, pomocí kterých mohou organizace modelovat svoje obchodní procesy. Jedná se pravděpodobně o světově nejpoužívanější standard pro modelování podnikových procesů. Jeho nespornou výhodou je, že narozdíl od hojně rozšířených vývojových diagramů, je BPMN standardizované, tudíž je možné modely vytvořené v BPMN automatizovat, je možné volně měnit nástroje, kterými jsou modely vytvářené a uživatelé tak nejsou závislí na jejich výrobcích. 

Za vznikem BPMN stála iniciativa BPMI (Business Process Management Initiative), jejíž primární motivací bylo vytvořit grafickou notaci, která bude srozumitelná všem účastníkům životního cyklu procesu (management, vývojáři, analytici) \cite{Vasicek2008}. Faktem je, že modely v BPMN v dnešní době slouží pro popis procesů na vysoké úrovni abstrakce, ale i pro popis těch nízkoúrovňových, které slouží jako podklad pro implementaci procesu v nějakém softwarovém nástroji.

Dalším cílem, se kterým bylo BPMN vytvořeno bylo ustanovit notaci, které umožní zobrazovat jednoduché i komplexní obchodní procesy \cite{Vasicek2008}, protože v té době obvyklé modelovací metody byly při vytváření rozsáhlých modelů velmi obtížně použitelné a vzniklé modely bylo složité udržovat.

O BPMN se dnes stará organizace Object Management Group (OMG). Za svojí popularitu vděčí BPMN, kromě výše zmíněných důvodů, zejména přístupnosti notace pro business uživatele, kteří jsou obeznámení s tradičními vývojovými diagramy, kterým se struktura diagramů i některé elementy v mnohém podobají \cite{Silver2011}. Rozdílem oproti vývojovým diagramům je ale již výše zmíněná standardizace použití jednotlivých e-mailů a tedy v tomto případě účelné omezení svobody uživatelů. Tento fakt umožňuje \textit{validovat} výsledné BPMN modely oproti specifikaci. Dalším rozdílem je možnost modelovat chování na základě výskytu nějaké definované události, což je situace, která se v reálném životě stává velmi často, ale ve vývojovém diagramu ji není možné vyjádřit. V neposlední řadě je v BPMN také možné modelovat komunikaci s entitami mimo organizaci nebo proces.

Jak tvrdí například \cite{Silver2011} v praxi vzniká velké množství \uv{špatného BPMN}, tedy modelů, které nejsou validní, kompletní nebo jednoznačné. Důvod je zřejmý – absence pevného teoretického základu, který by říkal více než k čemu který element z notace slouží a s kterým elementem je možné ho propojit. BPMN chybí \textit{metodologie}, která by přesně popisovala jak modely vytvářet a jak zaručit jejich konzistenci, jednoznačnost a kompletnost. V praxi tedy vidíme vznik metodologií, které nejsou součástí standardu BPMN, ale jsou adoptovány ve firmách právě z důvodu požadavků na výše zmíněné vlastnosti a rovněž pro zajištění kontinuity. Jednou z velmi rozšířených je metoda popsaná v publikaci \textit{BPMN Method and Style} vyvinutá Brucem Silverem, který se rovněž podílí na vývoji standardu BPMN. Cílem metody je jejím dodržováním vyvářet modely, které jsou

\begin{itemize}
\item korektní,
\item jednoznačné,
\item kompletní,
\item konzistentní.
\end{itemize}

\begin{quote}
\uv{Špatné BPMN} je dnes normou spíše než výjimečností. \cite{Silver2011}
\footnote{Bad BPMN is the norm rather than the exception. \cite{Silver2011}}
\end{quote}

\subsection{Verze 1.2 vs 2.0}
BPMN je v současnosti ve verzi 2.0, nicméně v praxi je stále hojně využívána i verze 1.2. Klíčovým rozdílem je standardizace převodu BPMN konstruktů do jazyka XML, což by umožnilo z grafického vyjádření modelu v notaci BPMN vygenerovat metamodel v jazyce XML, který by bylo možné automatizovat pomocí softwarových nástrojů. Taková řešení sice již existovala i u starších verzí BPMN, ale byla vždy závislá na interpretaci výrobce konkrétního BPMS řešení a tudíž jen obtížně přenositelná. Standard BPMN 2.0 by měl toto změnit. Co se týče grafických elementů, tak do verze 2.0 jich oproti 1.2 přibylo jen velmi málo a většina business uživatelů tak pravděpodobně ani rozdíl nepostřehne.

\section{Základní koncepty notace BPMN}
Dříve než přikročíme k popisu základních elementů notace BPMN

\subsection{BPMN diagram}
Diagram v BPMN není pouze grafickým vyjádřením podnikového procesu, ale zároveň vstupním budem pro \textit{sémantický model} v jazyce XML, který je (pokud to modelovací software umožňuje) vytvářen zároveň s grafickým diagramem aniž by do toho uživatel musel jakkoli zasahovat. Jak píše \cite{Silver2011} BPMN dovoluje existenci sémantického modelu bez jeho grafického vyjádření, ale ne naopak – sémantický model tedy musí vždy existovat.

Procesní model v BPMN neříká nic o tom, jak jsou jeho jednotlivé aktivity prováděny nebo proč jsou prováděny. Definuje pouze následující:

\begin{itemize}
\item \textit{pořadí} aktivit,
\item \textit{kdy} se aktivity provádějí,
\item za jakých \textit{podmínek} se aktivity provádějí.
\end{itemize}

\subsection{Aktivita v BPMN}
\cite{Silver2011} definuje \textit{aktivitu} v BPMN jako \textit{akci}, jednotku provedené práce. Aktivita je akce prováděná v rámci organizace opakovaně a je neměnná, neboli je vždy prováděna stejným způsobem a má jasně vymezený začátek a konec. Aktivita je v rámci modelu dále nedělitelná, tj. není možné ji rozložit na subaktivity.

\subsection{Proces v BPMN}
Samotný pojem \textit{proces} lze v BPMN jednoduše popsat jako posloupností aktivit z počátečního stavu do konečného stavu \cite{Silver2011}. \textit{Procesní model} je pak mapou všech možných cest – posloupností aktivit – z počátečního stavu do některého z konečných stavů. Podobně jako aktivita i proces je prováděn opakovaně a každá jeho instance musí být prováděna dle některé z cest definovaných v procesním modelu.

Dle \cite{Silver2013} má proces v BPMN 4 důležité aspekty:

\begin{itemize}
\item \textbf{Orchestrace} – jako orchestraci označujeme fakt, že procesy v BPMN se skládají z aktivit, které jsou vždy prováděny v určitém pořadí tak, jak je definováno v procesním modelu. Tyto aktivity jsou prováděny opakovaně a průběh jejich vykonávání má jasně definovaný začátek a konec. Procesní model navíc obsahuje všechny signifikatní možnosti, jak proces může proběhnout, ne jen jednu nejčastější nebo ad-hoc případy provádění.
\item \textbf{Participant} – samotný proces v BPMN je vnímán jako samostatná entita a je účastníkem \textit{spolupráce (collaboration)} s jinými entitami. Každý participant je unikátně propojen s jedním procesem.
\item \textbf{Množina vykonavatelů} – každá aktivita v BPMN má vykonavatele ač není v diagramu znázorněn. Pokud je aktivita součástí orchestrace popsané výše, je pak součástí procesu a to samé platí i pro jejího vykonavatele.
\item \textbf{Nezávislý actor} – BPMN používá z hlediska reálného života poměrně neobvyklou sémantiku, kdy u aktivity je tím, kdo požaduje její vykonání \textit{samotný proces} (jeho instance) a aktivitu provádí její vykonavatel (například zaměstnanec).
\end{itemize}

\subsection{Procesní logika}
\textit{Procesní logika} definuje všechny signifikatní možnosti, jak může proces probíhat (sekvence aktivit) od začátku do konce. Každý procesní model by měl obsahovat kompletní procesní logiku, pokud pro vypuštění některých možných cest neexistuje vážný důvod. Častým problémem modelů, které \cite{Silver2011} označuje jako \uv{špatné BPMN} je popis pouze jedné cesty (obvykle té \uv{šťastné}, tzv. \textit{happy flow}) a ignorování případů, které z nějakého důvodu probíhají odlišně.

\section{Základní elementy notace BPMN}

Standard BPMN ve verzi 2.0 obsahuje již více než 100 symbolů. Popis všech je mimo rozsah této práce. Nicméně v této sekci budou popsány všechny elementy, které populární publikace \cite{Silver2011} označuje jako \textit{Level 1} a několik vybraných elementů z množiny, které stejný autor označuje jako \textit{Level 2}, které se nám budou hodit později v dalších částech práce.

Symbolika použitá v BPMN je odvozená od klasické a hojně rozšířené flowchartovací techniky, která je intuitivní na pochopení i pro pozorovatele, který není obeznámen s problematikou modelování podnikových procesů.

Jak uvádí \cite{Silver2011} a \cite{Vasicek2008} pro základní práci obvykle stačí pouze základní druhy elementů, které jsou:

\begin{enumerate}
\item Počáteční událost (Start event)
\item Konečná událost (End event)
\item Aktivita (Activity)
\item Sekvenční tok (Sequence flow)
\item Tok zpráv (Message flow)
\item Brána (Gateway)
\item Bazén (Pool)
\item Plavecká dráha (Swimming lane)
\item Asociace (Association)
\item Datový objekt (Data object)
\item Datové úložiště (Data store)
\item Datová asociace (Data association)
\end{enumerate}

\subsection{Aktivita}


\section{Základní pravidla modelování v BPMN}


\section{Možnosti automatizace BPMN modelů}

While BPMN-I is specific to non-executable BPMN, the BPMN Implementer’s Guide also includes a section on executable BPMN, beginning with what that phrase means in the context of the BPMN 2.0 standard.  We’ll show how process data is represented in the BPMN XML and how it is mapped to variables, task I/O parameters, gateway conditions, message payloads, event definitions, service interfaces, and human task assignment rules in “executable BPMN”.  The basic structure has not changed since the first edition of the book, but the XML schema has changed significantly.  The new edition describes the proper serialization of BPMN 2.0 execution-related details in accordance with the final BPMN 2.0 specification, and relates that to the way these details are defined in real BPMN 2.0-based process automation tools.

BPMN originated in 2002 as the visual design layer of a new type of “transactional workflow” system from a consortium called BPMI.org, led by a startup named Intalio. Leveraging the distributed standards-based architecture of the web and web services, this new type of BPM would be a radical break from the proprietary workflow systems of the client/server era. One key difference would be making the process execution language a vendor-independent standard. As it developed the language, called BPML, BPMI.org reached a peak of 200 members, essentially all the major software vendors except IBM and Microsoft. Another difference would be business “In a nutshell,” recalls BPMI’s founder Ismael Ghalimi, “it would allow less-technical people to build transactional applications by drawing simple flow charts.”[6] BPML would not be coded by hand but generated from a diagram, which would also be standardized: BPMN. Existing process diagramming standards like UML were rejected as too IT-centric. BPMI demanded something more business-friendly. Howard Smith and Peter Fingar fleshed out the promise of business-empowerment through BPMN in a seminal 2002 book, BPM: The Third Wave, which correctly predicted that empowering business people to manage their own processes was critical to the evolution of BPM. BPMI.org produced a spec for BPMN 1.0 in 2004. Ghalimi continues: “Among [the BPMI members were] very many process modeling tool vendors who loved the idea of a standard notation for processes, and very many workflow vendors who hated the idea of a standard language for executing them. The former understood that they could provide a lot of value around the core process modeling tool. The latter knew all too well that fragmentation of the market helped preserve the status quo…” As it turned out, no one had anything to fear from BPML.

IBM and Microsoft countered with BPEL, a slightly different language layered on top of the new web services standard called WSDL. In an instant those two vendors trumped 200, and BPML was effectively wiped out.

\subsection{BPMN a BPEL}
Standard WS-BPEL popisuje význam a vlastnosti jednotlivých aktivit, jejich XML definici, ale nespecifikuje jejich grafický zápis.
Někteří dodavatelé proto vytvořili svoji vlastní grafickou notaci jazyka BPEL, která se s každou verzí zdokonaluje tak, aby byly procesy co nejpřehlednější.
Naproti tomu BPMN (Business Process Modeling Notation) je grafická notace pro modelování procesů názornou a přehlednou formou.

Cílem BPMN je poskytnout firmám možnost zmapovat procesy a pohlížet na ně v grafické podobě. Přitom podoba a smysl všech symbolů je standardizován,každý přesně rozumí tomu, co daný model znamená a jak proces funguje. To je obrovský přínos, pokud potřebujete procesy komunikovat napříč firmou, mezi odděleními, mezi obchodními a IT útvary.


Rozdíly mezi BPEL a BPMN jsou tedy zřejmé:

BPMN je popisná notace pro business modely procesů a je ustálená na úrovni grafických symbolů, jejich významů a logických propojení.
BPEL je standard pro provádění a popis procesů a hovoří spíše technickou řečí. Je standardizován na úrovni XML definice procesu. \cite{BPELcz} %todo přepsat vlastními slovy%


\nocite{*}
\bibliographystyle{plain}
\bibliography{Bibliography}

\end{document}