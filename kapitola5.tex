\documentclass[]{article}
\usepackage[czech]{babel}
\usepackage[utf8]{inputenc}
\usepackage{float}
\usepackage{graphicx}

\newcommand{\ptheory}{$\Psi$-theory}

\begin{document}

\title{Kapitola 5: Využití metodologie DEMO pro vytváření BPMN modelů}
\author{Bc. Štěpán Heller}
\date{\today}
\maketitle

\section{Motivace}
Tato kapitola se zabývá přístupy, jak využít silných stránek metodologie DEMO pro vytváření BPMN modelů. Cílem těchto přístupů je vytvořit modely podnikových procesů v BPMN, které budou splňovat stejné požadavky, jaké klade metodologie DEMO. Tedy, že modely musí být ucelené, konzistentní, úplné, výstižné a obsahují jen nezbytné množství informací. Jak píše \cite{VanNuffel2009}, BPMN začíná tam, kde DEMO končí, tedy by se mohlo zdát, že je vhodné jejich silné stránky spojit.

\subsection{Slabé stránky DEMO}
Jak již bylo v průběhu této práce několikrát zmíněno, na metodologii DEMO je nejcennější, že říká tvůrcům procesních modelů \uv{jak} by měli při vytváření modelů podnikových procesů postupovat a nedává jim tedy pouze notaci, pomocí které je možné to realizovat. Nutno však zároveň říct, že teoretický základ metodologie DEMO (tedy kombinace enterprise ontology a \ptheory) je pro běžného uživatele velmi nepřístupný a obtížný na pochopení, protože jeho základy sahají až k fundamentálním filozofickým konceptům, jakými právě ontologie jsou. 

Křivka učení je tedy v tomto případě velmi pozvolná a k plnému pochopení metodologie DEMO je potřeba hodně času a zkoušení metodou pokus-omyl. Autor této práce má zkušenosti z účasti na kurzu MI-MEP (Modelování ekonomických procesů) na Fakultě informačních technologií ČVUT, kde po dobu jednoho semestru bylo certifikovanými DEMO profesionály vyučováno DEMO po teoretické i praktické stránce. Z autorových zkušeností i po absolvování celého 3 měsíčního kurzu měli jeho účastníci problém se základními technikami metodologie DEMO, jakou je například korektní aplikace distinction axiomu performa-informa-forma analýzou, ale i s dalšími fundamentálními věcmi.

Nebudeme tedy daleko od pravdy, když konstatujeme, že v případě běžných uživatelů (návrhářů, analytiků, vývojářů) v rámci podniků bude problém podobný. Dalším nedostatkem DEMO je množství nástrojů, které DEMO podporují, kterých je minimum a žádný z nich dosud nepokrývá všechny aspect modely, které DEMO obsahuje. Nutno však říct, že komunita kolem DEMO je velmi aktivní a na vývoji těchto nástrojů se intenznivě pracuje.

Cílem této práce je umožnit takovým uživatelům nadále využívat nástroje, které znají (BPMN) a umí používat, ale v rámci metodologie DEMO identifikovat \uv{nezbytné minimum} z teoretického základu, které zvýší kvalitu výsledných procesních BPMN modelů a tím umožní i jejich lepší analýzu a optimalizaci

\subsection{Slabé stránky BPMN}
U BPMN nacházíme slabiny tam, kde jsme u DEMO indentifikovali silné stránky a naopak. Jak již bylo naznačeno v kapitole 2 velkým problémem BPMN je absence pevného teoretického základu, který dává návrhářům pevný rámec, podle kterého se mohou řídit a vytvářet modely, podle pevně daných pravidel a postupů. Absence takového pevného rámce pak ústí v to, že modely jsou neúplné a chybí v nich podstatné informace nebo v to, že v momentě, kdy různí uživatelé modelují jeden a ten samý proces, tak skončí s různými modely i když jsou všechny dle specifikace BPMN korektní.

Problematické je i absence seznamu pravidel, co který element v BPMN přesně vyjadřuje, kdy ho používat a kdy ne. Popis jednotlivých elementů je rozprostřen na 500 stranách v celé specifikaci BPMN \cite{Silver2011} a uživatelé jsou tak odkázáni na interpretaci této specifikace, která pochází většinou od tvůrců modelovacího nástroje, který tito uživatelé aktuálně používají. To pak dále tříští způsob, jakým jsou elementy používány.

V praxi v organizacích často vidíme, že mezi procesními analytiky a vývojáři přirozeně vzniká jakási pseudo-metodologie, právě z důvodu toho, aby bylo možné uvnitř organizace konzistentně vytvářet modely podnikových procesů stejným způsobem i v týmu více lidí a aby byla zajištěna kontinuita i v případě fluktuace členů tohoto týmu. Jedním z přínosů této práce, by mělo být vytvoření pevného rámce založeného na ontologiích a \ptheory, který by mohl nahradit tyto pseudo-metodologie.

Zajímavá zjištění uvádí \cite{Caetano2012}, kde při svém výzkumu aplikovali zásady, na kterých stojí metodologie DEMO, na dva klíčové procesy ve velké organizaci (více než 2000 zaměstnanců). Tyto procesy obsahovaly dohromady přes 500 aktivit a více než 60 aktorů. Analytici prověřovali, zda je procesní model konzistentní, tedy zda pořadí kroků v tomto procesu odpovídá transakčnímu axiomu. Dále se zajímali, zda je takový model kompletní, neboli jestli všechny kroky transakce dle transakčního axiomu DEMO se dají namapovat na aktivity BPMN modelu. Po důkladné analýze těchto procesů došli k následujícím zjištěním:

\begin{itemize}
\item V původních procesech chybělo 25 \% p-faktů neboli tyto procesy obsahovaly aktivity, které nebylo možné přiřadit k vytváření nějakého výsledku (hmotného či nehmontého).
\item Chybělo rovněž 25 \% request c-actů. Tedy výsledky byly produkovány bez explicitního požadavku na jejich tvorbu a tedy nebylo možné jasně identifikovat iniciátora celé transakce.
\item Chybělo 50 \% promise c-actů. Requesty byly implicitně potvrzovány bez jakékoliv jasné dohody mezi aktory.
\item Chybělo 25 \% state c-actů. Výsledky transakce tak nebyly jasně komunikovány s jejím inicátorem a tedy není jasné, zda dohlídnutí na výsledek transakce leží na iniciátorovi nebo exekutorovi.
\item Chybělo 40 \% accept c-actů. Nebylo tedy možné identifikovat, jak jsou výsledky transakce akceptovány jejím iniciátorem a jestli takové výsledky splňují požadavky, které na ně byly kladeny při requestu.
\end{itemize}

Je tedy zřejmé, že absence pevného teoretického je reálným problémem (ač ho uživatelé sami nepociťují \cite{VanNuffel2009}) a jeho vytvoření by zvýšilo kvalitu vytvářených modelů podnikových procesů.

\section{Benefity kombinace DEMO a BPMN}
Hlavním přínosem této práce je vytvoření rámce, který vychází z metodologie DEMO a slouží ke stanovéní základů metodologie, která v budoucnu umožní vytváření kvalitnějších BPMN modelů. Slovo \uv{kvalitnější} si na tomto místě zaslouží hlubší rozebrání.

V tuto chvíli není cílem této práce vytvořit metodologii, která bude dokonale kopírovat DEMO a v podstatě jen umožní vytvoření DEMO modelů za použití notace BPMN. Abstrahujme zde od toho, zda je to vůbec možné, to musí být potvrzeno dalším výzkumem. Pokud bychom se k takovému přístupu rozhodli, nedosáhli bychom vyřešení problémů uvedených v předchozí sekci %todo odkaz%
. Cílem této práce je nalézt takový teoretický rámec, který bude stále pochopitelný pro uživatele a umožní jim vytvářet kvalitnější BPMN modely, ale nebude s sebou nést dlouhou křivku učení, kterou obsahuje DEMO. Co je tedy míněno \uv{kvalitnějšími} BPMN modely?

\begin{itemize}
\item Modely takto vytvořené budou kompletní, tedy nebude chybět žádný krok dle transakčího axiomu.


dát sem ti různé přístupy k využití BPMN a DEMO dohromady???
\nocite{*}
\bibliographystyle{plain}
\bibliography{Bibliography}

\end{document}
