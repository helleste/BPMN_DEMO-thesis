\documentclass[]{article}
\usepackage[czech]{babel}
\usepackage[utf8]{inputenc}
\usepackage{float}
\usepackage{graphicx}


\begin{document}

\title{Kapitola 2: Techniky modelování podnikových procesů}
\author{Bc. Štěpán Heller}
\date{\today}
\maketitle

\section{Procesní model a důvody pro jeho tvorbu}
Ať už člověk vytváří jakýkoliv model, jeho cílem je zachytit nějaký jev, který je potřeba kvůli své komplexnosti zobrazit zjednodušenou vizuální formou, která bude pochopitelná i pro jiné lidi než je sám tvůrce modelu. Umět jev zachytit ve formě modelu je jedním z prvních kroků na cestě k tomu tento jev upravovat.

Přeneseno do světa podnikových procesů je to velmi podobné. Jedním z hlavních důvodů, proč organizace přistupují k práci s BPM je potřeba procesy upravovat a zejména optimalizovat. Aby to bylo možné, je potřeba nejdřív stanovit metriky a tyto metriky být pak schopen měřit. Základem pro všechny tyto kroky je ale korektní procesní model, který proces věrně popisuje.

\subsection{Definice procesního modelu}
Základní definice procesního modelu podle 
\begin{quote}
Procesní model je konceptualizací podnikového procesu v organizaci.
\footnote{Process model is a conceptualization of the (business) process in an enterprise. \cite{Dietz2006}}
\end{quote}
Čtenářsky přístupnější definici pak nabízí \cite{Recker2009}
\begin{quote}
Procesní model popisuje, většinou grafickou formou aktivity, události, jejich pořadí a propojení, které utváří podnikový proces.
\footnote{Process model describe, typically in a graphical way, the activities, events and control flow logic that constitues a business process.}
\end{quote}

\section{Základní techniky}
V této sekci si popíšeme populární techniky pro tvorbu procesních modelů.
\subsection{Vývojový diagram (flowchart)}
Vývojový diagram je pravděpodobně nejpopulárnější technikou pro modelování podnikových procesů. Vděčí za to zejména své jednoduchosti, dostupnosti mnoha nástrojů, které tuto techniku podporují a také její velké srozumitelnosti, která jí činá velmi snadno uchopitelnou i pro uživatele v organizaci, kteří nejsou příliš obeznámeni s problematikou modelování podnikových procesů.

\subsubsection{Základní pravidla}
Vývojové diagramy se skládají z několika málo základních symbolů. Tyto symboly se nazývají: \cite{Chytil2005}

\begin{itemize}
\item Startovací a ukončovací symboly – používají se pro vyznačení začátku a konce procesu
\item Šipky – zobrazují tzv. \uv{řídící tok}, tedy přechod v čase mezi jednotlivými symboly
\item Dílčí kroky procesu – jsou reprezantovány obdelníkem.
\item Podprogramy – zobrazeny obdelníkem se svislými čarami po stranách. Používají se pro zobrazení skupiny kroků procesu pomocí jediného symbolu.
\item Vstupy a výstupy – zobrazují tok informací směrem dovnitř i vně procesu, Pro jejich reprezentaci se používají lichoběžníky respektive rovnoběžníky.
\item Podmíněný cyklus – zobrazuje událost opakující dokud je splněna jasně definovaná podmínka. Zobrazuje se pomocí šestiúhelníku.
\item Podmíněný výraz – kosočtvercem je symbolizováno rozhodnutí a určuje tedy místo, kde dochází k větvení procesu.
\item Spojovací symbol – inverzním symbolem ke kosočtverci je ve vývojovém diagramu kruh, který se používá ke spojení více toků do jednoho.
\end{itemize}

\begin{figure}[H]\centering %todo překreslit obrázek%
\includegraphics[width=1.0\textwidth]{obrazky/flowchart}
\caption{Nákupní proces pomocí vývojového diagramu}
\label{fig:Flowchart}
\end{figure}

\subsubsection{Výhody a nevýhody}
Nespornou výhodou vývojových diagramů je právě jejich přístupnost pro uživatele a velmi strmá křivka učení, což dělá z této techniky první volbu pro případy, kdy je potřeba velmi rychle vymodelovat nějaký proces a organizace nemá zavedeny sofistikovanější metody BPM. Vývojové diagramy umožňují efektivnější komunikaci o problému v rámci týmu. 

Největší přednost vývojových diagramů je zároveň jejich největší slabinou. Právě přílišná jednoduchost této techniky dělá z modelování komplexnějších procesů poměrně komplikovanou a nepřehlednou záležitostí. Ve vývojových diagramech je také složitější modelovat některé jevy, jako například tzv. \uv{unhappy paths} a další nestandardní události, která však v životě procesů nastávají poměrně běžně. U vývojových diagramů je také obtížné dělat změny, protože to často vyžaduje kompletní překreslení celého diagramu.

\subsubsection{Použití}
Vývojové diagramy mají mnoho využití. Hodí se například pro komunikaci mezi organizací a jejími externími zákazníky, protože se dá předpokládat, že se s vývojovými diagrami už v minulosti setkali a budou jim tedy rozumět. Vhodné je také použít vývojový diagram v dokumentaci k softwaru nebo jinému systému, kterou budou číst různorodé skupiny uživatelů.

\subsection{BPMN}
BPMN nebo rozepsaně Business Process Modelling Notation je v současnosti de facto standardem na poli modelování podnikových procesů. Jeho využití je široké od IT přes obchod až například po komplexní dopravní systémy. Za svou popularitu vděčí zejména kombinaci dvou věcí, z nichž jedna je stále poměrně jednoduchá notace, která neobsahuje přehršel symbolů.

S trochou nadsázky by se dalo říct, že BPMN je vlastně rozšířením vývojového diagramu. Je určitě pravdou, že se BPMN touto jednoduchou technikou v mnohém inspirovalo a na jejích základech postavilo notaci, která umožňuje poměrně jednoduše modelovat i komplexní podnikové procesy a zároveň si stále uchovává dobrou srozumitelnost pro uživatele.

\subsubsection{Základní pravidla}
Jelikož se BPMN budeme podrobně věnovat v kapitole 3 %todo odkaz%
nemá smysl na tomto místě zacházet do přílišných detailů. Zatím si vystačíme s tím, že v BPMN jsou zásadními objekty aktivity, události, brány, počáteční a koncové symboly a \uv{šipky} neboli symboly řízení toku procesu. Vzhledově se příliš neliší od korespundujících symbolů ve vývojovém diagramu.

\begin{figure}[H]\centering %todo překreslit obrázek%
\includegraphics[width=1.0\textwidth]{obrazky/bpmn_nakupniproces}
\caption{Nákupní proces pomocí BPMN}
\label{fig:BPMN_nakupniproces}
\end{figure}

\subsubsection{Výhody a nevýhody}
Mezi hlavní výhody BPMN určitě můžeme zařadit fakt, že BPMN je standard, tedy je přesně definované jak a kdy, které symboly používat. O BPMN se stárá organizace OMG (Object Management Group) a kontinuálně pracuje na jeho rozvoji. Díky širokému rozšíření BPMN existuje na trhu velké množství placených i neplacených nástrojů, které umožňují modelování podnikových procesů pomocí této notace.

Další neoddiskutovatelnou předností je srozumitelnost notace, která je vysoká právě díky své podobnosti k vývojovým diagramům. Bez nutnosti zdlouhavého studia dokumentace je BPMN modelu schopen porozumět člověk z managementu společnosti a stejně tak i softwarový inženýr nebo vývojář. Právě pro ty ukrývá BPMN další výhodu a tou je poměrně přímočará převoditelnost BPMN modelů do strojově čitelných formátů, jako je jazyk XML nebo na něm založený jazyk BPEL. %todo ověřit%

Ačkoliv je používání notace BPMN definované v dokumentaci, reálné procesy v organizacích mohou být obtížně modelovatelné bez hluboké znalosti BPMN a může tedy docházet k vytváření nekorektních modelů nebo více různých modelů toho samého jevu. Dalším problémem dle \cite{Polancic2014} je, že někteří výrobci softwaru pro modelování v BPMN si tento standard ohýbají podle sebe nebo ho \uv{obohacují} o vlastní \uv{vylepšení}, která pak dělají takto vytvořené modely obtížně přenositelnými.

\subsubsection{Použití}
Organizace OMG, která BPMN nyní spravuje, uvádí jako hlavní poslání BPMN přenositelnost procesních modelů vytvořených v této notaci nezávisle na tvůrcích konkrétního modelovacího nástroje. BPMN je vhodné pro modelování podnikových procesů v celé jejich šíři.

\subsection{BPEL}
BPEL neboli Business Process Execution Langauge (a správněji WS-BPEL Web Service Business Process Execution Language) je v našem výčtu jedinou technikou pro modelování podnikových procesů, který nemá grafickou reprezentaci. Je to totiž jazyk. Své využití nachází zejména při automatizaci podnikových procesů. BPEL je založen na XML a v podstatě standradizuje definici podnikových procesů právě pomocí XML. \cite{BPELcz}

\subsection{UML}
UML neboli Unified Modeling Language je velmi populární grafický jazyk, zvláště v oblasti IT a vývoje softwarových systémů. Jak píše \cite{Eriksson} právě s tímhle cílem bylo UML původně také vytvořeno. Jenže jeho popularita se rychle rozšířila i do světa byznysu a UML přestalo dostačovat potřebám svých uživatelů. Proto bylo postupně rozšiřováno o další aspekty, které pokrývaly modelování podnikových procesů v celé jejich šíři.

UML obsahuje standardizovaný mechanismus jak jazyk rozšiřovat tak, aby jeho obecné principy mohly být doplněny o další vyhovující specifickému účelu, jako je například právě modelování podnikových procesů. Proto byl již v době vzniku UML vytvořen standardní profil pro modelování podnikového procesu \cite{Repa2007}. Tento profil pracuje zejména s Diagramem tříd a s Diagramem Use-Case. Diagram Use-Case je v tomto profilu používán pro zobrazení podnikových procesů a jejich interakce s aktéry a zákazníky. Oproti tomu Diagram tříd se používá spíše k zobrazení vnitřní struktury popisované organizace. Faktem je, že se standardní profil pro modelování podnikového procesu v praxi příliš neujal, snad kvůli své přílišné snaze podnikové procesy přiblížit k IT a informančím systémům \cite{Repa2007}.

UML se přesto pro modelování podnikových procesů používá a to buď v neformální podobě, kdy je organizacemi používán především Diagram aktivit, který je pro modelování podnikových procesů vhodný. Activity Diagram totiž umožňuje sekvenční i paralelní zobrazování aktivit a objekty na vstupu i výstupu procesu a také závislosti mezi jednotlivými aktivitami.

Pro větší přenositelnost a potřeby standardizace UML jazyka pro modelování podnikových procesů však vznikla celá řada rozšíření třetích stran. Mezi nejpoužívanější pak patří rozšíření podle H. Erikssona. \cite{Repa2007,Eriksson2000}. 

\begin{quote}
Erikssonův přístup je nejenom rozšířením UML, ale do značné míry plnohodnotnou metodou modelování procesů – určuje sadu modelů a diagramů, postavených vesměs na standardních diagramech UML. \cite{Repa2007}
\end{quote}

Erikssonovo rozšíření obsahuje více různých diagramů, přičemž pro potřeby samotného modelování podnikového proceus je nejzásadnější Diagram procesů, který je rozšířením právě již výše zmíněného Diagramu aktivit. Právě tomu se tedy budeme v této kapitole věnovat.

\subsubsection{Základní pravidla}


\subsubsection{Výhody a nevýhody}
\subsubsection{Použití}

\subsection{Petriho sítě}
\subsection{DEMO}
\section{Srovnání technik}

\nocite{*}
\bibliographystyle{plain}
\bibliography{Bibliography}

\end{document}