\documentclass[]{article}
\usepackage[czech]{babel}
\usepackage[utf8]{inputenc}
\usepackage{float}
\usepackage{graphicx}

\newcommand{\ptheory}{$\Psi$-theory}

\begin{document}

\title{Kapitola 4: Metodologie DEMO}
\author{Bc. Štěpán Heller}
\date{\today}
\maketitle

\section{O DEMO}
\subsection{Rozdíl mezi notací a metodologií}
V celém textu se objevují 2 jevy. O BPMN mluvíme jako o notaci a o DEMO jako o metodologii. Takto tyto techniky označujeme záměrně a myslím, že je na tomto místě účelné si vysvětlit rozdíl mezi pojmy \textit{notace} a \textit{metodologie}. Pochopení této odlišnosti totiž vnáší trochu světla k lepšímu pochopení rozdílu mezi BPMN a DEMO.

\subsubsection{Notace}
Notace označuje formální prostředky pro popis reality. Například právě v oblasti analýzy a modelování podnikových procesů je notací sada grafických objektů, které pak používáme pro popsání samotného procesu. Na notaci je obvykle navázána související metodika. \cite{Notace}

Metodikou nazýváme popis pracovního postupu nějaké činnosti, který je více či méně formalizovaný. \cite{Metodika} V případě modelovací techniky by taková metodika tedy popisovala jak při modelování postupovat a jak a kdy jednotlivé elementy přesně používat. To však v případě BPMN neplatí, jelikož BPMN není svázané s žádnou metodikou \cite{Vasicek2008} a tedy pokud se někde vyskytuje označení BPMN jako metodiky, je tato formulace chybná.

\subsubsection{Metodologie}
Metodologie je oproti tomu vědní disciplína, která se zabývá tvorbou metod a jejich aplikací. Metodologie vědy je tedy naukou o metodách. Jak píše \cite{Ochrana2009} je teorií k výběru výzkumných metod a návodem, jak vybrané metody (metodu) používat ve vědeckém zkoumání.

Pod vlivem angličtiny se však i v češtině často setkáváme s tím, že pojmy metodika a metodologie splývají a jsou často zaměňovány. Můžeme nicméně vidět, že rozdíl BPMN a DEMO je na první pohled zřejmý v tom, že notace BPMN nemá za sebou zdaleka tak robustní základ jako metodologie DEMO. Tento fakt má několik důsledků týkajících se přístupnosti a přímočarosti použití obou technik. Tyto důsledky budou ještě v této práci dále rozebrány.

\subsection{Motivace k vytvoření DEMO}
Jak shrnuje \cite{Vejrazkova2013}, u základní úvahy tvůrců DEMO byl současný stav podniků a organizací, které jsou velmi komplexní a z toho důvodu je velmi obtížné mít pomocí současných nástrojů jasnou představu o tom, jak přesně fungují a co se v nich děje.

Moderní organizace jsou totiž založeny na propojení sociálních a technických komponent, které spolu vzájemně komunikují. Komunikce je tedy nejdůležitějším aspektem celé metodologie DEMO. Dle \cite{Dietz2006} je ontologie podniků (Enterprise Ontology) nejvhodnějším prostředkem k pochopení konstrukce a operací v podniku.

\begin{quote}
DEMO bylo vytvořeno jako metodologie pro vytváření ontologického modelu podniku. \cite{Vejrazkova2012}
\footnote{DEMO was developed to be a methodology for creating an ontological model of an enterprise. \cite{Vejrazkova2012}}
\end{quote}

\section{Ontologie}

V úplně základním pojetí je ontologie definována jako nauka o bytí. Taková definice je samozřejmě pro čtenáře velmi abstraktní. Lepší by bylo ontologii definovat jako nauku o Bytí, tedy s velkým B, neboť ontologie se právě zabývá \uv{pouze} tím, co to znamená, že něco \uv{je}, jak to bytí vypadá a jak to funguje.

\begin{quote}
Ontologie (nebo ontologický model) organizace je definován jako porozumění chodu organizace, které je kompletně oproštěné od realizace a implementace vlastních činností.
\footnote{The ontology (or ontological model) of an enterprise is defined as an understanding of its operation, that is completely independent of the realization and the implementation of the enterprise. \cite{Dietz2005}}

Pro lepší pochopení toho, co ontologický model představuje je užitečné uvést rozdíl mezi teleologickým pojetím systému a ontologickým modelem systému.

Teleologický pohled na systém se zabývá funkcemi a službami, které systém poskytuje navenek. Teleologický model pak vypadá jako tzv. \uv{black-box model} neboli vidíte, že se vstupy změní na nějaké výstupy, ale už není vidět, jak k tomu došlo. Tento pohled (model) je vhodný pro užívání a řízení (věcí, systémů, organizací).

Onotlogický pohled se naopak zabývá tím, jak systém funguje uvnitř, tedy tím, \texit{jak} dojde k proměně vstupů na výstupy. Zabývá se tedy konstrukcí a chodem systému. Ontologický model je tedy typem tzv. \uv{white-box modelu}. Ontologický model najde uplatnění při budování a úpravách (věcí, systémů, organizací).

\subsection{Motivace pro zabývání se ontologiemi v organizaci}
Jak už bylo popsáno výše, ontologie v organizaci slouží zejména k porozumění jejímu chodu bez nutnosti zabývat se, jak jsou jednotlivé činnosti implementovány. Takový přístup je užitečný pro následující skupiny uživatelů: \cite{Dietz2005}

\begin{itemize}
\item \textbf{Manažeři} – Pro řízení větších celků je užitečné mít možnost oprostit se od detailů a dokázat se na chod takového celku podívat z vyšší perspektivy, tzv. \uv{big picture pohled}.
\item \textbf{Návrháři, inženýři, architekti} – Pro účely návrhu a úprav fungování chodu organizace je důležité mít podnikové procesy definované metodicky a nezávisle na jejich implementaci.
\item \textbf{Uživatelé} – Existují skupiny uživatelů (uvnitř i vně organizace), pro které je užitečné mít vhled i do fungování organizace nebo jejího celku.
\end{itemize}

\section{Teorie PSI ($\Psi$-theory)}

\begin{quote}
Teorie PSI neboli \ptheory  je teorie o fungování organizací. \cite{Dietz2005}
\footnote{The \ptheory is a theory about the operation of organizations. \cite{Dietz2005}}
\end{quote}

Zkratka PSI znamená \textit{Performance in Social Interaction}. Paradigma, na kterém je tato teorie založena říká, že subjekty, kterými jsou lidé v organizaci, vstupují do závazků a dodržují je. Tímto způsobem pak vzniká spolupráce mezi lidmi. %todo fuj%

Cílem \ptheory je umožnit porozumění funkcím (%todo funkcím?%)
organizace bez vlivu toho, jak jsou tyto funkce ve skutečnosti operativně vykonávány. Jak uvádí \cite{Vejrazkova2013} stejné cíle si klade i metodologie DEMO, takže je jen logické, že je právě na \ptheory postaveno. Porozumění této teorii je tedy nezbytné pro správné pochopení a používání DEMO.

Teorie PSI se skládá ze čtyř axiomů:

\begin{enumerate}
\item operační,
\item transakční,
\item kompoziční,
\item distinkční.
\end{enumerate}

\subsection{Operační axiom – The operation axiom}
První axiom \ptheory se nazývá operační. Jeho základem jsou dvě tvrzení: \cite{Dietz2006}:

\begin{enumerate}
\item The operation of an enterprise is constituted by the activities of actor roles, which are elementary chunks of authority and responsibility, fulfilled by subjects. %todo wtf!!! %
\item Při tom provádějí dva druhy činností: koordinační a produkční (\textit{coordination and production acts}). Výsledkem těchto činností jsou koordinační a produkční skutky (\textit{coordination and production facts}).
\end{enumerate}

Provádět produkční činnost znamená přivádět na svět něco nového a přispívat tak k podnikovým funkcím nebo službám. Produkční skutky pak mohou být hmotné i nehmotné. Příkladem těch hmotných může být například vyrobení pizzy, příkladem nehmotných zase napříkald vynsesení rozsudku soudem.

Provádět koordinační činnost znamená, že subjekty jednají v souladu se závazky k sobě navzájem, které se týkají tvorby produkčních skutků.

Kromě činností a skutků rozlišujeme ještě koordinační a produkční světy. Koordinační svět je možina koordinačních skutků a stejně tak produkční svět je množina produkčních skutků. Oba světy jsou tedy množinou skutků, které byly vytvořeny do konkrétního momentu v čase.

%todo obrázek operačního axiomu%

\subsubsection{Koordinační činnosti}
Koordinační činnost probíhá mezi dvěma subjekty z nichž jeden se nazývá vykonavatel (\textit{performer}) a druhý (\textit{adressee}). Koordinačních činností je několik typů, které můžeme rozdělit na intenční (\textit{intention}) a propoziční (\textit{proposition}).

Mezi příklady intenčních koordinačních činností řadíme:

\begin{itemize}
\item požadavek
\item příslib
\item dotázání
\item tvrzení
\end{itemize}

V případě propozičních koordinačních činností vykonavatel prohlašuje produkční fakt a příslušný čast, kdy má být proveden. %todo prohlašuje ne% 

%todo obrázek týhle sračky%

\subsubsection{Produkční skutky}
Jak již bylo naznačeno, produkční skutky jsou buď hmotné nebo nehmotné. Zde je nezbytné poukázat na to, kdy tyto skutky začnou v produkčním světě skutečně existovat. Před tím, než se to stane je totiž provést ještě dva koordinační skutky a to oznámení a přijetí. Teprve ve chvíli, kdy jsou tyto koordinační skutky provedeny začíná produkční fakt skutečně existovat v produkčním světě.

%todo doplnit sekci o actorech když bude malej rozsah%
\subssubection{Aktoři}
Aktoři jsou aktivní subjekty uvnitř organizace. Jednají autonomně, tedy jejich činnost není vyvolána nějakou událostí. \cite{Dietz2006} U aktorů existují tři důležité vlastnosti, kterými jsou kompetence, autorita a zodpovědnost.

Kompetencí je myšlena schopnost subjektu provádět produkční činnosti a souvesjící koordinační činnosti. \cite{Dietz2006} uvádí příklad instalatéra, který má znalosti a zkušenosti, které jsou nezbytné pro to být profesionálním instalatérem.

Aby mohl být subjekt schopen vykonávat určitou profesi, musí pro to mít nějaký autoritativní základ, jako například být zaměstnancem určité organizace a podobně.

Subjekt je vázán normami, které se vztahují k řečené autoritě nebo k obecným normám platným ve společnosti, které očekávají, že bude svojí autoritu vykonávat odpovědným způsobem. V příkladu instalatéra to znamená, že se očekává, že bude jednat zodpovědně se svými zákazníky.

\subsection{Transakční axiom – The transaction axiom}
Transkační axiom dále rozebírá produkční a koordinační činnosti a zejména to, jak spolu tyto činnosti souvisí. Základní myšlenkou transakčního axiomu, kterou formuluje \cite{Dietz2006} je, že koordinační činnosti probíhají postupně za sebou ve stejných vzorech. Tyhle vzory se nazývají transakce a vždy zahrnují dva aktory (iniciátor a vykonovatel) a jejich cílem je dosáhnout určitého výsledku, kterým je produkční skutek.

Každá transakce má tři fáze:

\begin{enumerate}
\item vyjednávací fáze (\textit{order phase}),
\item prováděcí fáze (\textit{execution phase}),
\item výsledková fáze (\textit{result phase}).
\end{enumerate}

V rámci vyjednávací fáze se iniciátor a vykonavatel snaží dojít k dohodě ohledně výsledku, kterého má být dosaženo (co, kdy). V prováděcí fázi je tento výsledek vytvořen a ve výsledkové fázi opět dochází k jednání mezi iniciátorem a vykonovatelem, jestli vytvořený výsledek odpovídá požadavku iniciátora.

Výsledek transakce (produkční skutek) začne existovat až ve chvíli, kdy je dokončena výsledková fáze, tedy produkční skutek je schválen a přijat iniciátorem transakce. Do tohoto okamžiku produkční skutek v našem výkladu neexistuje.

Transakční vzory rozlišujeme 2: základní a standardní.

\subsubsection{Zjednodušený transakční vzor}
Zjednodušený transakční vzor, jak už jeho název napovídá, je zjednodušený průběh transakce oproštěný od tzv. \textit{unhappy paths} neboli \uv{negativních scénářů}. Popisuje postup transakce v případě, kdy nenastanou žádné problémy, tedy vykonovatel vytvoří produkční skutek, který odpovídá požadavkům iniciátora a tento produkční skutek je tedy bez komplikací akceptován.

Průběh zjednodušeného transakčního vzoru:

\begin{enumerate}
\item Inicitátor formuluje \textit{požadavek} (\textit{request})
\item Vykonavatel učiní \textit{příslib} (\textit{promise})
\item Vykonavatel \textit{provede} požadavek (vytvoří produkční skutek) (\textit{execution})
\item Vykonavatel \textit{prohlásí} výsledek za hotový (\textit{state})
\item Iniciátor \textit{akceptuje} výsledek (\textit{accept})
\end{enumerate}

%todo obrázek%

\subsubsection{Standardní transakční vzor}
Standardní transakční vzor počítá i se situacemi, které v běžném životě neustále nastávají. Například se situací, kdy vykonavatel nemůže učinit příslib na požadavek iniciátora nebo iniciátor odmítne akceptovat výsledek transakce. Pokud toto nastane, dostává se celá transakce do tzv. \textit{diskusních stavů} (\textit{discussion states}), které mohou být buď \textit{nepřijmuto} (\textit{declined}) nebo \textit{odmítnuto} (\textit{rejected}).

Důvody pro odmítnutí \textit{požadavku} vykonovatelem výchazejí z těchto tří typů tvrzení (\textit{validity claims}):

\begin{itemize}
\item tvrzení pravdivosti (\textit{claim to truth}),
\item tvrzení oprávněnosti (\textit{claim to justice}),
\item tvrzení upřímnosti (\texit{claim to sincerity}).
\end{itemize}

\subsubsection{Odvolávací vzory}
V rámci standardního transakčního vzoru je možné odvolat kterýkoliv koordinační čin pomocí odvolávacího vzoru pro daný koordinační čin.

\subsection{Kompoziční axiom – The composition axiom}
Umět popsat strukturu konkrétní transakce je rozhodně přínosné, ale na ontologický popis organizace umět popsat jednotlivé transakce nestačí. Zde pak nastupuje kompoziční axiom, který se zabývá tím jak jsou jednotlivé transakce, respektive produkční skutky, propojené.

Kompoziční axiom tvrdí, že každá transakce je buď součástí jiné, je transakcí zákazníka organizace nebo je sebeaktivující (\textit{self-activated}). Logickou implikací tedy je, že transakce mohou obsahovat další transakce. \textit{Takto propojené transakce pak tvoří podnikovový proces.}

Jak píše \cite{Dietz2006} kompoziční axiom je tak základem pro definici podnikového procesu, která říká, že podnikový proces je množina volně propojených transakcí. 

%todo případně obrázky odvolávacích vzorů%

\subsection{Distinkční axiom – The distinction axiom}
Distinkční axiom tvrdí, že lidé mají tři různé typy schopností, které hrají roli v jejich chování.

\begin{itemize}
\item \textbf{forma} – jak už název napovídá, tato schopnost se týká formy v jaké jsou informace uchovávány, předávány, přijímány atd.
\item \textbf{informa} – v tomto případě jde o obsah informace a její komunikaci mezi lidmi a plně abstrahujeme od formy v jaké je informace komunikována.
\item \textbf{performa} – jedná se o nejvyšší formu lidských schopností. Jde zde o vytváření nových originálních věcí přímo nebo nepřímo pomocí komunikace. To se týká závazků, rozhodnutí, posuzování apod.
\end{itemize}

Jak píše \cite{Vejrazkova2013} pro jeden infologický čin (performa) musíme provést více infologických činů (informa) a pro jeden infologických činů musíme provést více datalogických činů (forma). Toto rozlišení umožňuje výrazně zjednodušit procesní modely, protože se při jejich tvorbě zabýváme pouze ontologickými činy.

\section{Teorém organizace – The organization theorem}

V předchozí sekci jsou popsány 4 axiomy \psitheory, které přináší z různých úhlů pohled na chod organizace, její operace a uspořádání. Tyto pojmy však samy o sobě nestačí k vytvoření ontologického modelu, který bude výstižný, úplný, ucelený a konzistentní. Teorém organizace se zabývá právě tím.

Dle teorému organizace je každá organizace strukturovaná jako heterogenní systém, který je tvořen třemi vrstvami, z nichž každá představuje jeden homogenní systém: \cite{Dietz2006}

\begin{itemize}
\item B-organizace (B=Business)
\item I-organizace (I=Intellect)
\item D-organizace (D=Document)
\end{itemize}

Mezi těmito vrstvami (systémy) existuje velká provázanost. D-organizace podporuje I-organizaci a I-organizace podporuje B-organizaci. Provázanost mezi těmito systému zajišťuje člověk. U tohoto konstatování je třeba se zastavit. Rozdělení organizace do tří systému si není možné představovat tak, že existují v organizaci nějaká B-oddělení, I-oddělení nebo D-oddělení nebo dokonce B-lidé, I-lidé či D-lidé. Naopak v realitě nic takového neexistuje. Lidi i skupiny lidí zastávají role ve všech systémech najednou a volně mezi nimi \uv{přechází}.

Rozdíl mezi jednotlivými systémy tvoří jejich výstupy. Jak uvádí \cite{Dietz2006} výstup B-organizace je ontologický, výstup I-organizace je infologický a výstup D-organizace je datalogický.

%todo obrázek%

Na vrcholu této pyramidy je B-organizace neboli ontologický level. Tím je naznačeno, že porozumění chodu organizace na této úrovni je kompletní, tedy není z něj nic vynecháno.

Pro lepší pochopení provázanosti jednotlivých systémů v organizaci bude nyní jejich provázanost hlouběji rozebrána. I-organizace poskytuje \uv{informační zdroje} pro fungování B-organizace. \cite{Dietz2006} uvádí příklad s výpočtem denního obratu, který B-aktor v B-organizaci chce vytvořit. Za tímto účelem je ale nejprve v I-organizaci I-aktorem třeba provést několik jasně definovaných výpočtů (I-transakce) a následně doručit B-aktorovi výsledek, kterým je právě denní obrat.

Dále je nutné rozebrat propojení mezi I-organizací a I-aktorem s D-organizaci a D-aktorem. V příkladu počítání denního obratu musí nejdříve I-aktor sčítající jednotlivé položky, které denní obrat tvoří, někde tyto položky (čísla) opatřit. Tyto údaje získá samozřejmě v D-organizaci pomocí D-transakcí.

Jak už bylo naznačeno výše, není důležité, jestli v tomto konkrétním případě je B-aktor, I-aktor a D-aktor několik osob nebo jeden člověk a stejně tak B-organizace, I-organizace či C-organizace mohou být na několika kontinentech nebo uvnitř jedné kanceláře.

\section{Metodologie}
V této sekci bude rozebrána vlastní metodologie DEMO. Rozebrány budou zejména dvě věci: typy modelů, ze kterých se DEMO skládá a doporučený postup, jak v DEMO modelovat. Pro přesný popis jednotlivých elementů je možné dohledat například v \cite{2006}.

Jak píše \cite{Vejrazkova2012}, základními elementy DEMO jsou \textit{ontologické transakce} a \textit{aktoři}.

\subsection{Ontologická transakce}
Ontologické transakce jsou transakce, jejichž výsledkem je vytvoření nčeho nového, ať už se jedná o věc hmotnou či nehmotnou. Jak už uvádí distinkční axiom \ptheory, který byl popsán výše, ontologické transakce probíhají na ontologické úrovni (performa).

\subsection{Aktor}
Každá transakce má vždy jednoho iniciátor a jednoho vykonavatele.

\subsection{Modely DEMO (Aspect models)}
Struktura a popis jednotlivých vzorů vychází zejména z \cite{Vejrazkova2003} a \cite{Dietz2005}.

Jak už bylo zevrubně popsáno v kapitole 2 %todo přidat odkaz%
, DEMO se skládá ze 4 hlavních modelů a dalších podmodelů (většinou diagramy nebo tabulky). Jedná se o:

\begin{itemize}
\item Construction Model (CM)
\item Process Model (PM)
\item State Model (SM)
\item Action Model (AM)
\end{itemize}

Každý z těchto modelů se pohybuje na jiné úrovni abstrakce, což se dá dobře popsat na principu pyramidy, který je naznačen na obrázku. %todo link na obrázek%
Construction Model, který je usazen na vrcholu této pyramidy pracuje s nejvyšší úrovní abstrakce a naopak Action Model, který je na obrázku zobrazen při základu pyramidy je už velmi podrobný a detailní. Všechny tyto modely dohromady tvoří kompletní ontologickou znalost organizace.

\subsubsection{Construction Model}
Jak už jeho název napovídá, Construction Model se zabývá konstrukcí organizace na té nejvyšší úrovni abstrakce. Construction Model tvoří dva podmodely: \textit{Interaction Model (IAM)} a \textit{Interstriction Model (ISM)}. Cílem Construction Modelu je v podstatě jen identifikovat jednotlivé ontologické transakce a jejich výsledky. 

\subsubsubsection{Interaction Model (IAM)}
Interaction Model (IAM) zobrazuje aktory (inicitátory a exekutory) a transakce, které tito aktoři vykonávají. Jedná se o vysoce abstraktní pohled, takže v něm není vidět jednotlivé kroky transakcí, ale čistě jen interakci mezi aktory, která je prováděna právě prostřednictvím transakce.

IAM se skládá z Transaction Result Table (TRT) a Actor Transaction Diagram (ATD). TRT je jednoduchá tabulka indetifikovaných ontologických transakcí a jejich výsledků. ATD je diagram, který ukazuje transakce mezi aktory.

\subsubsubsection{Interstriction Model (ISM)}
ISM je založen na IAM a obsahuje 2 diagramy a tabulku:

\begin{itemize}
\item \textbf{Actor Bank Diagram (ABD)} – zobrazuje vztah mezi aktory a informančími bankami
\item \textbf{Organization Construction Diagram (OCD)} – jedná se o kombinaci ABD s ATD. Tedy aktoři, transakce mezi nimi a navíc ještě propojení s informačními bankami.
\item \textbf{Bank Contents Table (BCT)} – tabulka popisující obsah bank faktů.
\end{itemize}

\subsubsection{Process Model (PM)}
Pokud se v Construction Model nacházel na vysoké úrovni abstrakce, tak Process Model jde o krok více do detailu. Kde CM použe pojmenovával transakce, PM rozvádí jejich jednotlivé kroky tak, jak jsou popsány v transakčním axiomu \ptheory. Zároveň ukazuje také vztahy mezi jednotlivými transakcemi. Stále je potřeba mít na paměti, že ačkoliv PM detailně popisuje strukturu podnikového procesu, tak je stále abstrahován od implementace a vlastní realizace daného procesu včetně takových věcí jako je výměna dat apod.

V případě přípravy na vývoj informačního systému je PM dobré místo, odkud začít se soupisem požadavků a případů užití.


PM se skládá z:

\begin{itemize}
\item \textbf{Process Structure Diagram (PSD)} – jak už název napovídá, zobrazuje strukturu procesu. Jak už bylo popsáno výše, podnikový proces se skládá z jedné či více vzájemně provázaných transakcí. Kroky, které nejsou popsány v PSD nejsou povolené a PSD by měl obsahovat i odvolávací vzory.
\item \textbf{Information Use Table (IUT)} – přímo se váže na State Model. Pro každou objektovou třídu, typ skutku a výsledku ze State Modelu. %todo ???%
\end{itemize}

\subsubsection{State Model (SM)}
State Model se zabývá především p-světem – jeho objektovými třídami, typy skutků, typy výsledků a existenčními ontologickými pravidly. SM je přímo navázaný na Action Model, zobrazuje ale pouze informace, které jsou relevantní pro chod organizace. SM obsahuje:

\begin{itemize}
\item \textbf{Object Fact Diagram (OFD)} – tento diagram zobrazuje vztah mezi objektovými třídami a typy výsledků.
\item \textbf{Object Property List (OPL)} – popisuje objektové třídy a jejich vlastnosti.
\end{itemize}

\subsubsection{Action Model (AM)}
Action Model existuje v metodologii DEMO na té nejnižší úrovni abstrakce a tudíž je velmi detailní a ostatní modely na něm stojí, AM popisuje především pravidla, kterými se operace v organizaci řídí. Tato pravidla jsou v AM popsána slovně pomocí pseudo-algoritmického jazyka, ve kterém specifikují co má být uděláno při požadavku, příslibu, prohlášení a akceptaci.

Už z popisu výše je zřejmé, že AM bude velmi užitečný nástroj při implementaci informačního softwaru. Action Model je ontologicky atomický, což znamená, že již nemůže být dále rozdělen na podmodely.


\nocite{*}
\bibliographystyle{plain}
\bibliography{Bibliography}

\end{document}